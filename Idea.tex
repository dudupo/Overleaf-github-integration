

\documentclass{article}
\usepackage[utf8]{inputenc}
\usepackage[a4paper, total={6in, 8in}]{geometry}
\usepackage{braket}
\usepackage{xcolor}
\usepackage{amsmath}
\usepackage{amsfonts}
\usepackage[english]{babel}
\usepackage{amsthm}
\usepackage{amssymb}

\newtheorem{theorem}{Theorem}[section]
\newtheorem{lemma}[theorem]{Lemma}
\newtheorem{definition}[theorem]{Definition}
\newtheorem{conjecture}[theorem]{Conjecture}
% \usepackage{biblatex} %Imports biblatex package

\usepackage[
backend=biber,
style=alphabetic,
sorting=ynt
]{biblatex}

\addbibresource{IdeaBib.bib}

\begin{document}

\newcommand{\commentt}[1]{\textcolor{blue}{ \textbf{[COMMENT]} #1}}
\newcommand{\ctt}[1]{\commentt{#1}}
\newcommand{\prb}[1]{ \mathbf{Pr} \left[ {#1} \right]}
\newcommand{\onotation}[1]{\(\mathcal{O} \left( {#1}  \right) \)}
\newcommand{\ona}[1]{\onotation{#1}}

\newcommand{\GG}{\tilde{G} }
\newcommand{\TGG}{\(\tilde{G}\) }
\newcommand{\Prb}[1]{ \mathbf{Pr}\left[ {#1} \right] }
\renewcommand\qedsymbol{$\blacksquare$}

\newcommand{\Htr}[1]{\tilde{H}_{#1}}
\newcommand{\Ht}{\Htr{l}}
\newcommand{\Htp}[2]{ \Htr{#1}\left(\ket{#2}\right)  }
\newcommand{\ps}{\ket{\psi}}
\newcommand{\hpst}{\left( H, \ps \right)}
\newcommand{\Ham}{Hamiltonian }
\section{Week qPCP Theorem.}

\paragraph{} \ctt{Work on progress.} 
\begin{enumerate}
    \item Add description of the 1-D, 2-local Hamiltonian.
    \item Add tickz plot of \(\Ht\).
    \item Add algorithm figure Lemma 1.4. prove correction, running time. and Lemma 1.5.  
\end{enumerate}

%%    each problem that it's solution can be verified      circuit the  Quantum  probabilistically checkable proofs. QMA, the class of that problems
\paragraph{Preamble.} The qPCP conjecture states that all the decision problems with polynomial-time verifiable proof also have a probabilistic proof that their verification requires measuring only a "few" random qubits from it. The classic theorem is proved by showing that if an optional witness invalidates more than one restriction (i.e., in the k-coloring problem), then there is a reduction to another problem in which the mapped witness invalidates a constant fraction of the restrictions. 

That process could be thought of as scattering a local error over the global domain.
\ctt{Add description of the 1-D, 2-local Hamiltonian.  }
Theorem: QMA-complete even when 2-local on a line of 8-dimensional qudits, i.e. when the
qudits are arranged on a line and only nearest-neighbour interactions are present [proven by Nagaj6 \cite{hallgren2013local}.

\begin{definition} Consider the an \textbf{1D}, 2-local Hamiltonian \(H = \frac{1}{m}\sum^{m}_{i=0}{H_i}\) over a linear chain of d-qudits.   Let \( \Ht:\mathcal{H} \mapsto \mathbb{R} \) for 
\( l \le m \) be the function, such for every state \( \ket{\psi} \in \mathcal{H} \), \begin{equation*}
    \Ht\left(\ket{\psi}\right) = \frac{1}{m}\braket{\psi|\sum^{l}_{i=0}{H_i}|\psi} 
\end{equation*}  . 
\end{definition}

\begin{definition} Let \(i \le j \le m \in \mathbb{N} \) such that \( \Htp{i}{\psi} = \Htp{j}{\psi} = 0 \) and for every \( t \in (i,j) \Rightarrow  \Htp{t}{\psi} \neq 0 \). , then the \([i,j]\)-segment will be said  \(\ps\)\textbf{-Positive Segment}. The length of the segment is \( |j-i|\). From the definition it's directly follows that \(\ps\) is the ground state of \(H\) iff there is integer set \( i_1 , i_2 ... i_k \) such that \(i_{j+1} < i_{j}\), \(i_{1} = 0\), \(i_{k} = m\) and each of the segments \( [i_{j},i_{j+1}] \) is a \(\ps\)-Positive Segment. We will call to such sets the \textbf{zeros} of \( \left( H, \ps \right) \).
\end{definition}

\begin{lemma} Let \(S\) be zeros set of \(\hpst \). If \(|S| \ge \sqrt{n} \) then there are at least \( \frac{\sqrt{n}}{2} \) \(\ps\)-positive segments in \(S\) at length less then \(2\sqrt{n}\). 
\end{lemma}

\begin{proof} By contradiction, denote by \( T \) the \( \ps\)-positive segments at length less then \( \sqrt{n}\) and assume \( |T|  < \frac{1}{2}\sqrt{n} \). Then the total length of the positive segments is \begin{equation*}
\begin{split}
    \sum_{[i_{j},i_{j+1}]:i_{j}\in S}{|i_{j+1}-i_{j}|} &= \sum_{[i_{j},i_{j+1}]\in T}{|i_{j+1}-i_{j}|}+\sum_{[i_{j},i_{j+1}] \notin T :i_{j}\in S}{|i_{j+1}-i_{j}|} > |T| + \left(\sqrt{n}-|T|\right)2\sqrt{n} \\ 
    &= 2n - 2\sqrt{n}|T| + |T| > 2n - 2\sqrt{n}|T| > n 
\end{split}
\end{equation*} contradiction.
\end{proof}

\begin{lemma} Consider \( \hpst \) with zeros set with size greater then \(\sqrt{n}\). Suppose that \(\ps \) is indeed the ground state of \(H\). Then there is a verification process that requires \ona{\sqrt{n}} time and measurements and succeeds with probability \( \Omega\left(\frac{1}{\sqrt{n}}\right)\).
\end{lemma}

\begin{proof}
Denote by \(Q\) a FIFO queue, choose a random qudit and insert it into \(Q\), define \(f_0 = 0 \). while \(f_0 \neq 0\) (and it's not the first iteration) pop the first qudit from \(Q\), denote it by \(q\) and apply all the Hamiltonian which act over it. Add the result to 
\begin{equation*}
     f_{i+1} \leftarrow f_{i} + \frac{1}{m}\sum_{H_{j} \text{ act on q} }{ \bra{q,p} H_i \ket{q,p} } 
\end{equation*} If the iteration number is greater then \(4\sqrt{n}\) then returns \textbf{False}, else if \(f_{i+1} =0 \) return \textbf{True} otherwise,repeat.       
\end{proof}

% \begin{lemma} Given Hamiltonian \( H \) decision problem, whether \(H\) has a ground state \( \ps \) such that the size of the zeros set of \(\hpst\) is greater then \(\sqrt{n} \), belongs qPCP.     
% \end{lemma}

\begin{conjecture}
Let \(H\) be an \Ham over qudits pair \( \mathbb{C}_d \times \mathbb{C}_d \) such that for every \((i,j) \in [d^{2}]\times[d^{2}]\) the entry \( |H_{ij}| \ge \varepsilon \) and let \( \ps \) be an arbitrary state.


\end{conjecture}

\printbibliography 

\end{document}