\documentclass{article}
\usepackage[utf8]{inputenc}
\usepackage[a4paper, total={6in, 8in}]{geometry}
\usepackage{braket}
\usepackage{xcolor}
\usepackage{amsmath}
\usepackage{amsfonts}

% \usepackage{biblatex} %Imports biblatex package

\usepackage[
backend=biber,
style=alphabetic,
sorting=ynt
]{biblatex}

\addbibresource{sample.bib} %Import the bibliography file

\newcommand{\commentt}[1]{\textcolor{blue}{ \textbf{[COMMENT]} #1}}
\newcommand{\ctt}[1]{\commentt{#1}}
\newcommand{\prb}[1]{ \mathbf{Pr} \left[ {#1} \right]}
\newcommand{\onotation}[1]{\(\mathcal{O} \left( {#1}  \right) \)}
\newcommand{\ona}[1]{\onotation{#1}}
% \newcommand{\cal}{\mathcal}

\newcommand{\eps}{\epsilon}

\title{CLH}
\author{David Ponarovsky}
% \date{Aug 2021}
\begin{document}
\maketitle

\paragraph{Definition 1} 
We say that such a graph is an $\epsilon$ small-set bi-partite
expander, if for any set $S$ of size at most $k$ particles, the number of local
terms incident on these particles is at least $D_R |S| (1-\epsilon)$, where
$D_R$ is the right degree of the graph.
We note that
$D_R |S|$ is the maximal possible number of constraints acting on those
particles, so $\epsilon$ can be viewed a ``correction'' to this number.

\paragraph{Fact 1}
Consider $S\subseteq R$  in a bi-partite graph $G(R,L:E)$
and let $S$ $\epsilon$-expanding, for $\eps<\frac{1}{2}$.
Then a fraction at most $2\eps$ of all
vertices of $\Gamma(S)$ have degree strictly larger than $1$ in $S$.

\paragraph{Fact 2}
Let $S\subseteq R$ in a bi-partite graph $G = (R,L;E)$, such that $S$ is $\eps$ expanding.
Then there exists a vertex $q\in S$,
such that the fraction of neighbors of $q$ with at least two neighbors in $S$
is at most $2\eps$.

\subsection{Commuting terms and \texorpdfstring{$C^*$}{TEXT}-algebras}\label{sec:bv}
We now state the lemma of Bravyi and Vyalyi \cite{Bra} precisely:
\begin{lemma}\label{lem:BV}
Let $H_i,H_j$ be two local terms on Hilbert space ${\cal H}$, $[H_i,H_j]=0$.
Let ${\cal H}_{int}$ denote the intersection of $supp(H_i)\cap supp(H_j)$, where $supp(H)$ is the subset of qudits examined non-trivially by $H$.
Then, there exists a direct-sum decomposition
$${\cal H}_{int} = \bigoplus_{\alpha} {\cal H}_{int}^{\alpha},$$
where for each $\alpha$ we have:
$${\cal H}_{int}^{\alpha} = {\cal H}_{int}^{\alpha,i}\otimes {\cal H}_{int}^{\alpha,j},$$
such that
both $H_i,H_j$ preserve all subspaces ${\cal H}_{int}^{\alpha}$, and moreover,
for each $\alpha$ $H_i|_{\alpha}$ is non-trivial only on the Hilbert space ${\cal H}_{int}^{\alpha,i}$,
whereas $H_j|_{\alpha}$ is non-trivial only on the Hilbert space ${\cal H}_{int}^{\alpha,j}$.
\end{lemma}




% \printbibliography 

\end{document}
